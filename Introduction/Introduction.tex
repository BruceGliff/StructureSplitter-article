\section{Введение}
\label{sec:Introduction}
Вычисления на видеоускорителях являются важной частью повседневной и научной жизни. Огромная часть работы выполняется на них с большой производительностью за счёт большого числа независимых ядер, способных выполнять параллельно небольшие вычисления. Тысячи вычислительных потоков видеоускорителей выполняют вычисления, достигая нескольких десятков и сотен терафлопс. Для достижения такой производительности критически необходимо использовать вычислительные ресурсы по максимуму. Часть обязанностей ложится на программиста, а часть на компилятор, чтобы максимально равномерно распределить нагрузку.

Одной из самых долгих операций при вычислениях является работа с памятью. Примерное распределение тактов выполнения инструкций от типа памяти для CPU следующее: обращение к регистрам непосредственное, обращение к L1-cache - 1 такт, L2-cache - 10 тактов, RAM - 200 тактов, HDDR - 2млн тактов. В видеускорителях обращение в память немного отличается, но тенденция будет сохраняться. Поэтому важно быть аккуратным при работе с памятью и обращаться к ней как можно меньше для достижения большей производительности.

Структуры - это набор из одной или более переменных, возможно различных типов, сгруппированных под одним именем для удобства обработки~\cite{Kern}. Они оказались полезными при организации сложных данных особенно в программах больше среднего, так как во многих ситуациях они позволяют сгруппировать данные таким образом, что с ними можно обращаться как с одним целым, а не как с отдельными объектами.

Так же в видеускорителях широко распространена концепция SIMD - Single Instruction Multiple Data. Операция производится над всеми значениями в векторе за раз, и каждое значение в векторе вычисляется независимо. В векторном backend компилятора intel уже существует стадия, векторизующая различные типы.

В этой статье мы представим алгоритм деления структур, позволяющий свести сложные структуры к более простым, которые можно будет разложить на регистры видеоускорителя. Так как структуры широко распространены в программах, поэтому любые действия и оптимизации над ними будут влиять на большое количество приложений.
