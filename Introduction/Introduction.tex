\section{Введение}
\label{sec:Introduction}
Вычисления на видеоускорителях являются эффективным средством решения большого класса практически важных задач. 
Существенный прирост производительности по сравнению с CPU достигается за счёт большого числа независимых ядер, способных выполнять вычисления параллельно.
% тут надо на что-то сослаться

Тысячи вычислительных потоков видеоускорителей совместно достигают нескольких десятков и даже сотен терафлопс. Но на практике производительность программы зависит от усилий по оптимизации.
Часть обязанностей здесь ложится на программиста, а часть, особенно связанная с низкоуровневыми вещами -- на компилятор.

Одной из самых долгих операций при вычислениях является работа с памятью. 
% тут надо сделать небольшую выжимку по GPU память и на что-то сослаться

Структуры - это набор из одной или более переменных, возможно различных типов, сгруппированных под одним именем для удобства обработки~\cite{Kern}.
Они полезны при организации сложных данных особенно в больших программах, так как они позволяют сгруппировать данные таким образом, что с ними можно обращаться как с одним целым, а не как с отдельными объектами.
Поскольку структуры широко распространены в программах, любые действия и оптимизации над ними будут влиять на большое количество приложений, что накладывает дополнительные требования по корректности и стабильности преобразований.

В видеускорителях Intel Xe на низком уровне реализована концепция SIMD - Single Instruction Multiple Data.
Широкая операция производится над всеми значениями в векторе за раз, и каждое значение в векторе вычисляется независимо.
В векторном backend компилятора Intel Graphics Compiler уже существует стадия, векторизующая различные типы данных.
% ссылка на статью Кена по CM

В этой статье мы представим алгоритм разбиения структур, позволяющий свести сложные структуры к более простым, которые можно будет разложить на регистры видеоускорителя.
% TODO: дописать что мы ещё сделаем: представим алгоритм, сделаем замеры...